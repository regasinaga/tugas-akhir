%% This document is created by 
%%  Dr. Putu Harry Gunawan
%% Template untuk Proposal TA 1 dan TA
%% Template ini digunakan untuk penulisan proposal TA 1 atau TA Fakultas Informatika, Telkom University.

\documentclass[a4paper,12pt,oneside]{book}
\usepackage[utf8]{inputenc}
\usepackage{sectsty}
\usepackage{graphicx}
\usepackage{epstopdf}
\usepackage{algorithm}
\usepackage{algpseudocode}
\usepackage{array}
\usepackage[table]{xcolor}
\usepackage{anysize}
\usepackage{amsmath}
\usepackage{url}
\usepackage{amssymb}
\usepackage[bahasa]{babel}
\usepackage{indentfirst} %Spasi untuk paragraf pertama
\usepackage{geometry}
\marginsize{4cm}{3cm}{3cm}{3cm} %{left}{right}{top}{bottom}
\usepackage[compact]{titlesec} 
\usepackage{etoolbox}
\usepackage{longtable}
\usepackage{verbatim}

\usepackage{listings}
\usepackage{color}
\usepackage{adjustbox}
\usepackage{multirow}

\definecolor{lightgray}{rgb}{.9,.9,.9}
\definecolor{darkgray}{rgb}{.4,.4,.4}
\definecolor{purple}{rgb}{0.65, 0.12, 0.82}

\makeatletter
\patchcmd{\ttlh@hang}{\parindent\z@}{\parindent\z@\leavevmode}{}{}
\patchcmd{\ttlh@hang}{\noindent}{}{}{}
\makeatother

\chapterfont{\centering}
\newcommand{\bigsize}{\fontsize{16pt}{14pt}\selectfont}
\chapterfont{\centering\bigsize\bfseries}
\sectionfont{\large\bfseries}
\usepackage{tikz}
\usetikzlibrary{shapes.geometric, arrows}
\renewcommand{\thechapter}{\Roman{chapter}}
\renewcommand\thesection{\arabic{chapter}.\arabic{section}}
\renewcommand\thesubsection{\thesection.\arabic{subsection}}
\renewcommand{\theequation}{\arabic{chapter}.\arabic{equation}}
\renewcommand{\thefigure}{\arabic{chapter}.\arabic{figure}}
\renewcommand{\thetable}{\arabic{chapter}.\arabic{table}}

\renewcommand\bibname{Daftar Pustaka}
\addto{\captionsbahasa}{\renewcommand{\bibname}{Daftar Pustaka}}
\usepackage{fancyhdr}
\pagestyle{fancy}
\lhead{}
\chead{}
\rhead{}
\lfoot{}
\cfoot{\thepage}
\rfoot{}
\renewcommand{\headrulewidth}{0pt}

\makeatletter

%%%%%%%%%%%%%%%%%%%%%%%%%%%%%%%%%%%%%%%%%%%%%%%%%%%%%%%%%%%%
%
%  Berikut adalah data-data yang wajib diisi oleh mahasiswa
%
%%%%%%%%%%%%%%%%%%%%%%%%%%%%%%%%%%%%%%%%%%%%%%%%%%%%%%%%%%%%

\title{Implementasi \textit{Context-Aware Recommender System} Berbasis \textit{Ontology} untuk Merekomendasikan Tujuan Wisata di Bandung Raya}\let\Title\@title   %Judul dalam bahasa Indonesia

\newcommand{\EngTitle}{Context-aware Recommender System Implementation Based on Ontology for Recommending Tourism Destination at Greater Bandung}  %Judul dalam bahasa Inggris

\author{Rizaldy Hafid Arigi}  \let\Author\@author  %Nama mhs
\newcommand{\NIM}{1103130256}
\newcommand{\Prodi}{Teknik Informatika}
\newcommand{\KK}{Modelling Computational Experiment} %UNTUK TA
\newcommand{\Gelar}{Sains Komputasi} % UNTUK TA
\date{2016}           \let\Date\@date %Maskkan hanya tahun saja
\newcommand{\Tanggal}{11} % Tanggal Pengesahan
\newcommand{\Bulan}{November} % Bulan Pengesahan
\newcommand{\PembimbingSatu}{Z.K. Abdurrahman Baizal, S.Si.,M.Kom}
\newcommand{\NIPPembimbingSatu}{99750047}
\newcommand{\PembimbingDua}{Anisa Herdiani, S.T.,M.T}
\newcommand{\NIPPembimbingDua}{15850002}
\newcommand{\Kaprodi}{Dr. Ketua Program Studi}
\newcommand{\NIPKaprodi}{95650581-1}
\newif\iflogTA
\logTAtrue   
%%%%%% WARNING kode ini diaktifkan untuk format TUGAS AKHIR
%%%%%% dinonaktifkan jika hanya digunakan untuk Proposal TA 1
\makeatother
\linespread{1}


\begin{document}
\pagenumbering{roman} 
%%\maketitle
\begin{titlepage}
\thispagestyle{empty}
%\vspace*{0.7cm}
\include{Cover}
\pagebreak
\thispagestyle{empty}
\include{Lembar-Persetujuan}
\pagebreak
\end{titlepage}
\addcontentsline{toc}{chapter}{Abstrak}
\chapter*{Abstrak}

Sektor pariwisata menunjukkan tren kenaikan sumbangan devisa negara. Salah satu faktor kenaikan tersebut adalah semakin mudahnya mendapatkan informasi tujuan wisata di Indonesia melalui layanan Internet yang ada seperti TripAdvisor dan Instagram. Namun hingga saat ini, masih belum ada layanan Internet komersil yang mampu memberikan rekomendasi tujuan wisata yang terpersonalisasi. \textit{Recommender system} adalah teknik dan kakas perangkat lunak yang digunakan untuk menyediakan rekomendasi item yang sesuai dengan kebutuhan pengguna. Untuk dapat menyediakan rekomendasi item sebagai keluaran sistem, tentu dibutuhkan masukan yang sesuai. Salah satu teknik yang dapat diterapkan untuk memperoleh masukan \textit{recommender system} adalah \textit{context-aware}. \textit{Recommender system} dapat bersifat \textit{context-aware} dengan memanfaatkan informasi kontekstual pengguna seperti lokasi dan cuaca.

Tugas akhir ini disusun melalui pengembangan dan pengujian \textit{recommender system} berbasis \textit{mobile} yang bersifat \textit{context-aware} dengan memanfaatkan masukan berupa lokasi pengguna, cuaca di sekitar pengguna, waktu lokal dan preferensi kategori wisata, kemudian diproses dengan melakukan \textit{reasoning} pada salah satu model objek dalam \textit{knowledge-based}, yaitu \textit{ontology}, sehingga menghasilkan rekomendasi tujuan wisata.

\vspace{0.5 cm}
\begin{flushleft}
{\textbf{Kata Kunci:} \textit{recommender system}, \textit{context-aware}, \textit{mobile}, \textit{knowledge-based}, \textit{ontology}}
\end{flushleft}
\iflogTA
\pagebreak
\addcontentsline{toc}{chapter}{Abstract}
\chapter*{Abstract}

\par
Tourism sector shows increasing contribution for country revenue. One of key factor is how convenient to get Indonesia tourism information through internet service such as TripAdvisor
and Instagram. However, those commercial internet services are not yet able to provide personalized tourism destination
recommendation.
\par
Recommender systems are software tools and technique providing suggestions for item recommendation
based on user's needs. To provide item recommendation as system output, indeed suitable system input is needed. One of technique that can be implemented
to get recommender system input is context-aware.
\par
This undergraduate thesis is written through recommender system development and testing which has context-aware properties. This recommender
system is making use of user location, weather around tourism destinations, local time and user preference towards 
tourism destination category. Those system inputs are processed using knowledge-based ontological reasoning and recommender system
will propose tourism destination recommendation.

\vspace{0.5 cm}
\begin{flushleft}
{\textbf{Keywords:} tourism destination, recommender system, context-aware, knowledge-based, ontology}
\end{flushleft}
\pagebreak
\addcontentsline{toc}{chapter}{Lembar Persembahan}
\chapter*{Lembar Persembahan}

Alhamdulillah, setelah melalui perjalanan yang panjang akhirnya tugas akhir ini dapat selesai. Selama
pengerjaan tugas akhir ini, penyusun banyak dibantu dengan doa, motivasi dan panduan dari berbagai
pihak. Pada kesempatan ini, penyusun ingin mengucapkan terima kasih kepada:

\begin{enumerate}
	\item Allah SWT, yang telah memberikan rahmat dan karunia-Nya kepada penyusun sehingga dapat 
	menyelesaikan tugas akhir ini. 
	\item Kedua orang tua yang saya cintai, mama dan ayah yang selalu memberikan semangat dan dukungan.
	Semoga Allah tetap melimpahkan berkah kepada mereka. 
	\item Pak Baizal selaku pembimbing I dan bu Annisa selaku pembimbing II yang telah mendengarkan, 
	membimbing dan memberikan panduan dikala penyusun menemui kesulitan. Terima kasih pak Baizal dan
	bu Annisa, semoga sukses selalu.
	\item Pak Dana Kusumo selaku dosen wali, terima kasih atas masukan dan informasi yang diberikan kepada penyusun.
	Penyusun tidak akan lupa jasa bapak.
	\item Kawan-kawan yang senantiasa mengisi "markas" Hardware and Embedded System Studio. Zavli, Fiqih, Bayu, 
	kak Ari, kak Dick, kak Andra fatass, kak Kris, kak Niken. Kalian luar biasa!
	\item Teman-teman satu kelas IF-37-02, yang selalu bersama-sama dari awal dan tetap bersama hingga akhir.
	Penyusun tidak akan melupakan kalian.
\end{enumerate}
\pagebreak
\addcontentsline{toc}{chapter}{Kata Pengantar}
\chapter*{Kata Pengantar}

\par
Assalamualaikum wr, wb.
\newline
\par
Alhamduillah, puja dan puji syukur saya panjatkan ke hadirat Allah SWT yang telah memberikan rahmat dan
karunia-Nyan sehingga penyusun dapat menyelesaikan tugas akhir dengan judul "\Title".
\par
Dalam tugas akhir ini, penyusun bertujuan untuk mengimplementasikan dan menguji \textit{recommender system}
berbasis \textit{ontology} yang mampu merekomendasikan tujuan wisata dengan memanfaatkan lokasi
pengguna, waktu lokal, cuaca di tujuan wisata dan prefensi pengguna terhadap kategori wisata tertentu
sebagai masukan untuk sistem. Penyusun berharap sistem yang dikembangkan dapat bermanfaat untuk menjadi
sebuah purwarupa untuk \textit{recommender system} yang akan dikembangkan di masa mendatang.
\par
Semoga tugas akhir ini bermanfaat bagi siapapun yang membacanya.

\vspace{2 cm}
\begin{flushright}
Bandung, \today
\\[2cm]
\Author
\end{flushright}
\pagebreak
\fi
\cleardoublepage
\addcontentsline{toc}{chapter}{Daftar Isi}
\tableofcontents
\iflogTA
\newpage
\cleardoublepage
\addcontentsline{toc}{chapter}{Daftar Gambar}
\listoffigures
\newpage
\cleardoublepage
\addcontentsline{toc}{chapter}{Daftar Tabel}
\listoftables
%\pagebreak
\fi
%
\cleardoublepage
\pagenumbering{arabic}
\chapter{Pendahuluan}
\section{Latar Belakang}

Pariwisata telah menjadi sektor ekonomi yang penting di Indonesia. Hal tersebut terbukti ketika pada tahun 2014, sektor pariwisata menyumpang penerimaan devisa negara terbesar
keempat setelah komoditas minyak, batu bara dan kelapa sawit. Pada tahun 2009, sektor pariwisata bahkan menempati peringkat ketiga melebihi batu bara\cite{bps1}. Meskipun memang peringkat penyumbang devisa negara turun, sektor pariwisata berhasil meningkatkan sumbangan devisa dari USD 6,297 miliar di tahun 2010, hingga menjadi USD 11,16613 miliar di tahun 2014 \cite{bps2}.
\par
Pada era teknologi informasi ini, mendapatkan informasi umum apapun yang diinginkan adalah hal yang mudah. Calon turis mencari tahu informasi tujuan wisata secara manual dengan \textit{search engine} atau media sosial merupakan hal yang sudah biasa. Layanan di Internet yang sangat populer untuk mendapatkan rekomendasi tujuan wisata diantaranya TripAdvisor dan Instagram. Namun, mendapatkan informasi tujuan wisata yang sesuai dengan preferensi pengguna melalui cara ini memiliki kelemahan, diantaranya adalah menghasilkan rekomendasi yang kurang terpersonalisasi.
\par 
Rekomendasi yang kurang terpersonalisasi terjadi karena sistem terlalu menitikberatkan pada tujuan wisata yang sangat populer atau sistem tidak tanggap dengan kondisi yang dihadapi pengguna. Untuk mengatasi hal ini, maka \textit{recommender system} yang digunakan harus bersifat \textit{context-aware}\cite{alhazbi2013} dan menggunakan teknik \textit{knowldge-based} untuk memberikan rekomendasi yang adil. Salah satu model representasi pada \textit{knowledge-based} adalah \textit{ontology}. \textit{Ontology} memiliki keunggulan yaitu dapat merepresentasikan informasi kontekstual secara semantik dan kemudahan untuk menjelaskan bagaimana sistem dapat mencapai suatu kesimpulan. Namun, \textit{ontology} sendiri tidak dapat melakukan \textit{reasoning} dengan sendirinya karena \textit{ontology} hanya merepresentasikan pengetahuan, karena itu dibutuhkan metode \textit{reasoning} untuk mengambil kesimpulan dari representasi pengetahuan yang ada.
\par
Pada prakteknya, \textit{context-aware} cocok untuk diterapkan pada domain pariwisata. \textit{Recommender system} yang bersifat \textit{context-aware} dapat mendukung turis yang sedang bergerak dengan memberikan rekomendasi tujuan wisata yang cocok berdasarkan keadaan lingkungan turis berada dan preferensi turis tersebut\cite{alhazbi2013}. Hal itu dapat dicapai dengan memanfaatkan lokasi pengguna, cuaca dan waktu saat itu \cite{jakkhupan2015}. 
\par
Pada tugas akhir ini akan dilakukan pengembangan dan pengujian \textit{recommender system} dengan teknik \textit{knowldge-based} dan bersifat \textit{context-aware} dengan memanfaatkan variabel lokasi pengguna, cuaca dan waktu serta preferensi kategori wisata yang ingin dikunjungi. Semua variabel masukan tersebut diproses dengan metode \textit{reasoning} berdasarkan model \textit{ontology} yang dibangun dan menghasilkan keluaran berupa rekomendasi tujuan wisata. Pengujian sistem dilakukan dari sisi akurasi rekomendasi dan persepsi pengguna.

\section{Perumusan Masalah}
\begin{enumerate}
	\item Bagaimana model \textit{ontology} yang tepat untuk merepresentasikan pengetahuan sistem pada domain pariwisata?
	\item Bagaimana metode \textit{reasoning} untuk merekomendasikan tujuan wisata?
	\item Bagaimana skema pengujian dari sisi akurasi dan persepsi pengguna untuk \textit{recommender system} yang bersifat \textit{context-aware}?
	
\end{enumerate}
\section{Tujuan Penulisan}
\begin{enumerate}
	\item Membangun model \textit{ontology} yang tepat untuk mereprsentasikan pengetahuan sistem pada domain pariwisata.
	\item Mengimplementasikan metode \textit{reasoning} untuk merekomendasikan tujuan wisata.
	\item Menguji performansi sistem dari sisi akurasi (melibatkan \textit{domain expert}) dan persepsi pengguna.
\end{enumerate}
\section{Batasan Masalah}
\begin{enumerate}
	\item Sistem hanya merekomendasikan tujuan wisata di Bandung Raya, yang meliputi: Kota Bandung, Kabupaten Bandung, Kabupaten Bandung Barat dan Kabupaten Sumedang.
	\item Data masukan hanya berupa lokasi pengguna, cuaca  dan waktu di lokasi pengguna, serta preferensi kategori tujuan wisata pengguna.
	\item Sistem hanya mengeluarkan keluaran berupa rekomendasi tujuan wisata.
	\item \textit{Mobile platform} yang digunakan adalah Android.
\end{enumerate}
\section{Rencana Kegiatan}
Kegiatan-kegiatan yang akan dilakukan selama proses penyusunan tugas akhir adalah sebagai berikut:
\begin{enumerate}
	\item Desain sistem
	\par
	Tahap ini menitikberatkan pada perancangan sisi \textit{front-end}, \textit{back-end}, dan representasi \textit{ontology}.
	\item Pengumpulan data lokasi wisata di Bandung.
	\par
	Mengumpulkan data-data wisata di Bandung dengan metode observasi dan wawancara.
	\item Implementasi sistem
	\par
	Mengimplementasikan desain yang telah dibuat pada tahap desain sistem.
	\item Pengujian sistem.
	\par
	Melakukan dua jenis pengujian, satu bertujuan untuk mencari \textit{bug} dan mendapatkan akurasi rekomendasi tujuan wisata dari domain ahli serta sisi 
	persepsi pengguna. 
	\item Pembuatan laporan.
	\par
	Tahap penyusunan laporan berdasarkan kaidah penulisan tugas akhir yang berlaku di institusi.
\end{enumerate}
\section{Jadwal Kegiatan}
Berikut adalah tabel kegiatan dan bulan pelaksanaan.
 
\begin{table}[h!]

\begin{tabular}{||c c c c c c c c||} 
 \hline
 No & Kegiatan & Januari & Februari & Maret & April & Mei & Juni \\ [1ex] 
 \hline\hline
 1 & Desain sistem & \cellcolor{blue!25} & \cellcolor{blue!25} & &&&\\ 
 2 & Pengumpulan data tujuan wisata & & \cellcolor{blue!25} & &&&\\ 
 3 & Implementasi & & \cellcolor{blue!25} & \cellcolor{blue!25} & \cellcolor{blue!25} &&\\
 4 & Pengujian & & & & & \cellcolor{blue!25} &\\
 5 & Pembuatan laporan & & & & & \cellcolor{blue!25} & \cellcolor{blue!25} \\ [1ex]

 \hline
\end{tabular}
\caption{Tabel rencana kegiatan}
\label{table:1}
\end{table}
%
\chapter{Kajian Pustaka}

\section{Wisata dan Pariwisata}
Wisata adalah kegiatan perjalanan yang dilakukan oleh seseorang atau sekelompok orang dengan mengunjungi tempat tertentu untuk tujuan rekreasi, pengembangan pribadi, atau mempelajari
keunikan daya tarik wisata yang dikunjungi dalam jangka waktu sementara.

Pariwisata adalah berbagai macam kegiatan wisata dab didukung berbagai fasilitas serta layanan yang disediakan oleh masyarakat, pengusaha, pemerintah dan Pemerintah Daerah\cite{uupariwisata}.

\section{Recommender System}

\textit{Recommender system} (RS) adalah kakas dan teknik yang digunakan untuk mendapatkan sebuah saran atau rekomendasi item yang dapat digunakan oleh pengguna \cite{ricci2011book}. Rekomendasi item merupakan sebuah keluaran dari proses pembuatan keputusan. Item dapat berupa barang apa yang sebaiknya untuk dibeli, musik apa yang cocok dengan selera pengguna, dan masih banyak lagi.
\par
RS telah banyak digunakan pada media sosial dan layanan-layanan multimedia, contohnya pada Youtube. Youtube adalah sebuah layanan berbagi video milik Google yang sangat populer. RS pada Youtube berperan untuk memberikan rekomendasi video berdasarkan aktivitas pengguna ketika menggunakan Youtube.
\par
Ada tiga teknik yang paling sering digunakan untuk membangun RS, diantaranya adalah \textit{content-based}, \textit{collaborative filtering} dan \textit{knowledge-based}. 
\par
Pada \textit{content-based} RS belajar untuk menghasilkan rekomendasi item kepada pengguna berdasarkan apa yang pengguna sukai sebelumnya\cite{moreno2013sigtur}. Ciri item akan dibandingkan dan dihitung untuk mengetahui asosiasi antar item. Contohnya jika pengguna memberi respon suka pada beberapa video mengenai komputer, maka RS akan memberikan rekomendasi berupa video yang berbicara mengenai pemrograman komputer.
\par
Sedikit berbeda dengan \textit{content-based}, \textit{collaborative filtering} akan menghasilkan rekomendasi item kepada pengguna aktif berdasarkan apa yang pengguna lain sukai sebelumnya yang memiliki selera sama dengan pengguna aktif tersebut\cite{castillo2008samap}. Kemiripan selera dihitung berdasarkan kemiripan histori antar pengguna.
Kedua pendekatan yang telah disebutkan memiliki kelemahan. Permasalahan teknik \textit{content-based} adalah \textit{cold start} dan \textit{overspecialization}. \textit{cold start} muncul ketika pengguna baru aktif di sistem dan RS belum memiliki data mengenai preferensi pengguna. Sedangkan \textit{overspecialization} adalah masalah ketika RS memberi rekomendasi yang kemiripannya terlalu dekat dengan item sebelumnya yang disukai pengguna\cite{blanco2008flexible}. \textit{Collaborative filtering} juga memiliki permasalahan \textit{cold start}\cite{ricci2011book}. Permasalahan \textit{cold start} dan \textit{overspecialization} tidak dimiliki oleh \textit{knowledge-based}.
\par
\textit{Knowledge-based} adalah teknik RS yang merekomendasikan suatu item berdasarkan suatu representasi pengetahuan bagaimana ciri item bisa memenuhi kebutuhan dan selera pengguna serta tingkat kegunaan item untuk pengguna. \textit{Knowledge-based}  menggunakan fungsi kesamaan yang mengestimasi seberapa cocok kebutuhan pengguna dengan rekomendasi yang akan diberikan tanpa memerlukan data histori apapun dari pengguna sehingga masalah \textit{cold start} dan \textit{over-specialization} tidak akan muncul. Selain tidak membutuhkan data histori, \textit{knowledge-based} cocok untuk digunakan untuk pengembangan RS berdasarkan kasus dunia nyata.

\section{Context-aware}

Dalam bidang \textit{ubiqiotous computing}, konteks merupakan lokasi pengguna, identitas orang-orang disekitar pengguna, objek disekitarnya dan perubahan pada elemen-elemen tersebut \cite{ricci2011book}. Informasi kontekstual adalah hal yang sangat vital dalam perolehan rekomendasi yang relevan. Hal tersebut menjadikan \textit{context-aware} merupakan salah satu pendekatan yang efektif dalam memahami kebutuhan pengguna.

Pendekatan penggunaan informasi kontekstual untuk proses rekomendasi dapat dikategorikan menjadi dua kelompok, yang pertama adalah rekomendasi yang menggunakan \textit{context-driven querying} dan yang kedua adalah \textit{contextual preference elicitation and estimation}\cite{ricci2011book}. Pendekatan pertama telah digunakan pada RS untuk pariwisata dan perangkat \textit{mobile}. Sistem menggunakan pendekatan ini untuk mengekstrak informasi kontekstual langsung dari pengguna \cite{van2004context}\cite{abowd1997cyberguide}, contohnya dengan memberi pertanyaan bagaimana suasana hati pengguna saat ini, atau berdasar informasi lingkungan pengguna seperti lokasi, cuaca dan waktu untuk melakukan eksekusi \textit{query} dan mendapatkan item yang paling cocok dengan pengguna. Sedangkan pendekatan kedua memodelkan dan mempelajari preferensi pengguna dengan \textit{collaborative filtering}, \textit{content-based}, atau teknik analisis data dari \textit{machine learning}. Pada RS yang penyusun akan kembangkan, RS menggunakan pendekatan pertama karena pendekatan pertama cocok untuk RS dengan paradigma \textit{knowledge-based} \cite{ricci2011book} dan kompleksitas komputasi yang lebih kecil dari pendketan kedua.

\section{Semantic Web dan Ontology}

\textit{World Wide Web Consortium} (W3C) mendefinisikan \textit{semantic web} sebagai kerangka kerja umum yang dapat memperkenankan data untuk bisa diberikan dan digunakan ulang di lintas aplikasi, perusahaan dan batasan komunitas. Secara khusus, teknologi \textit{semantic web} menawarkan kemampuan menghubungkan data pada sebuah jaringan.
\par
\textit{Ontology} merupakan salah satu bagian penting dalam domain \textit{semantic web}. \textit{Ontology} adalah sebuah cara formal untuk merepresentasikan jaringan taksonomi dan klasifikasi serta mendefinisikan struktur pengetahuan. Pada tugas akhir ini, \textit{ontology} akan digunakan untuk merepresentasikan pengetahuan kategori wisata. \textit{Ontology} akan diimplementasikan dengan menggunakan bahasa khusus untuk membangun \textit{ontology} yaitu \textit{Web Ontology Language} (OWL).
\par
Beberapa penelitian telah mengajukan suatu metode untuk merekonstruksi \textit{ontology}. Salah satunya adalah \textit{Methontology}\cite{jones1998methodologies}. Tahap-tahap yang dilakukan dalam \textit{Methontology} adalah:
\begin{enumerate}
	\item Spesifikasi \par Mengidentifikasi tujuan \textit{ontology}, ruang lingkup \textit{ontology}, termasuk istilah didalamnya. keluaran dari tahap ini adalah spesifikasi \textit{ontology} dalam bahasa manusia.
	\item Akuisisi pengetahuan \par Wawancara dengan ahli dapat membantu akuisisi pengetahuan secara signifikan.
	\item Konseptualisasi  \par Data kunci pada suatu domain seperti konsep, \textit{instance}, relasi kata kerja, sifat direpresentasikan secara informal (bahasa manusia).
	\item Integrasi  \par Tahap untuk menentukan apakah \textit{ontology} yang dibangun perlu diintegrasikan dengan \textit{ontology} lain yang sudah ada.
	\item Implementasi \par \textit{ontology} direpresentasikan secara formal pada bahasa tertentu, contohnya SPARQL. 
	\item Evaluasi \par Tahap untuk mencari inkonsistensi dan redundansi. Inkonsistensi yang dimaksudkan adalah suatu pengetahuan yang dapat diinferensikan berdasar definisi dan aksioma lain yang hampir tidak ada hubungan. Sedangkan redundansi dapat terjadi pada kelas, subkelas atau \textit{instance}.
	\item Dokumentasi
\end{enumerate}


\section{Android}
Android adalah sistem operasi untuk perangkat \textit{mobile} yang dikembangkan oleh Google dengan memanfaatkan \textit{kernel} dari sistem operasi Linux. Sistem operasi ini telah banyak digunakan pada \textit{smartphone}, tablet pc, TV, \textit{wearable device}, dan masih banyak lagi.

Android digunakan sebagai \textit{platform} untuk sisi \textit{front-end} karena kepopuleran Android yang dapat dilihat dari perolehan \textit{market share} untuk pasar \textit{smartphone}\cite{androidmarket}.

\section{Google Maps API}
Google Maps API \textit{(Application Programming Interface)} adalah sebuah \textit{interface} khusus yang dapat digunakan pengembang perangkat lunak sehingga aplikasi yang dikembangkan dapat menggunakan layanan,fitur dan basis data pada Google Maps.
\par
Google Maps API dipilih karena kemudahan integrasi API pada pengembangan aplikasi Android.  

\section{OpenWeatherMap API}
OpenWeatherMap adalah sebuah API yang menyediakan layanan untuk mendapatkan data cuaca terkini. OpenWeatherMap API memiliki dukungan untuk berbagai macam \textit{platform} seperti Android dan iOS.
\par
Pada tugas akhir ini, OpenWeatherMap akan digunakan untuk mendapatkan data cuaca terkini di lokasi pengguna.

\section{Jython}
Jython merupakan bahasa pemrograman Python yang terintegrasi dengan Java. Berbeda dengan Python biasa yang berjalan dengan \textit{C compiler}, Jython berjalan dengan menggunakan
Java Virtual Machine (JVM). 

\section{Protokol HTTP}
\textit{Hypertext Transfer Protocol} (HTTP) adalah protokol aplikasi yang menjadi dasar komunikasi data pada \textit{World Wide Web}. \textit{Hypertext} adalah teks dengan struktur tertentu yang dikirim diantara \textit{node}. HTTP menjadi protokol agar bisa mengirim \textit{Hypertext}.
\par
HTTP berfungsi untuk memfasilitasi permintaan dan respon data pada model \textit{client-server}. Pada contoh dunia nyata, \textit{web browser} berperan sebagai \textit{client} dengan mengirimkan pesan permintaan HTTP ke \textit{server} yang berperan untuk \textit{hosting}. \textit{Server} akan menyediakan sumber daya yang disebutkan berdasar permintaan HTTP yang diterima dan mengirim respon HTTP kepada \textit{client}.
\par
HTTP memiliki beberapa metode permintaan seperti \textit{GET}, \textit{POST}, \textit{DELETE}, \textit{CONNECT} dan \textit{TRACE}. Metode permintaan yang paling sering digunakan adalah \textit{GET} dan \textit{POST}. Pesan yang dikirim dalam permintaan berisi baris permintaan, \textit{header} dan badan pesan (opsional). Sedangkan respon yang dikirim dengan HTTP berupa kode status permintaan, \textit{header}, dan badan pesan.
\par
Tugas akhir ini menggunakan HTTP sebagai perantara pertukaran pesan antara ponsel pengguna dengan proses yang berjalan pada \textit{server}. HTTP dipilih karena HTTP dapat mengirim pesan dalam berbagai tipe konten seperti teks biasa, html, atau Javascript Object Notation (JSON).   

\section{Javascript Object Notation}
\textit{Javascript Object Notation} atau yang biasa disebut JSON adalah sebuah format pertukaran data. JSON memiliki kelebihan dapat dipahami manusia dengan mudah dan kemudahan \textit{parsing} oleh mesin. 
\newline
Tugas akhir ini menggunakan JSON sebagai format pertukaran data antara \textit{client} dan \textit{web server}.

\section{SPARQL}
SPARQL adalah bahasa \textit{query} untuk basis data \textit{semantic}. SPARQL memiliki kapabilitas membaca dan memanipulasi data yang tersimpan dalam format \textit{Resource Description Framework}(RDF). Data dalam format RDF direpresentasikan dalam bentuk graf. Meskipun begitu, data RDF dapat dikategorikan sebagai database relasional SQL yang menggunakan tiga kolom yaitu subjek, predikat dan objek.
SPARQL akan digunakan untuk menyimpan data \textit{ontology} yang merepresentasikan pegetahuan sistem mengenai lokasi pariwisata Bandung. 
 
%
\chapter{Metodologi dan Desain Sistem}

\section{Pengembangan Ontology}
Dari sepuluh tahap \textit{Methontology}, terdapat tiga tahap utama yang menghasilkan keluaran formal 
yaitu \textit{specification}, \textit{conceptualization} dan \textit{implementation}.
\par

\subsection{Specification}
Dalam tahap \textit{specification} pengembangan \textit{ontology}, terdapat tiga hal utama yang yang harus dideskripsikan yaitu domain, tujuan dan lingkup \textit{ontology}
yang dikembangkan.
\par
Tahap ini akan menghasilkan dokumen spesifikasi kebutuhan \textit{ontology} pada domain pariwisata. Tabel 3.1 adalah spesifikasi dari \textit{ontology} yang akan dibangun.

\begin{table}[h]
\begin{center}
\begin{tabular}{ |c|m{12cm}| } 
\hline
	Domain & Pariwisata \\
	\hline
	Tujuan & Ontology untuk merepresentasikan pengetahuan klasifikasi tujuan wisata dan aktivitas yang dapat dilakukan\\
	\hline
	Lingkup & Daftar kategori tujuan wisata, daftar tingkat kedekatan suatu tujuan wisata dengan kategori tujuan wisata, daftar kategori cuaca, daftar tujuan wisata,
	daftar atribut tujuan wisata. \\ 
	\hline
	Sumber & Wawancara dengan pakar. \newline 
	buku \textit{Tourism Planning, An Integrated and Sustainable Development Approach} karya Edward Inskeep. \\
	\hline
\end{tabular}
\end{center}
\caption{Tabel spesifikasi kebutuhan ontology yang akan dibangun}
\label{table2}
\end{table}

Domain dari \textit{ontology} yang dibuat berada di pariwisata, meliputi tujuan wisata dan kategorinya. Untuk bisa mengetahui bagaimana tujuan wisata dikategorikan,
maka perlu adanya konsultasi dengan pakar terkait dan studi pustaka pada buku rujukan yang direkomendasikan oleh pakar.

\subsection{Conceptualization}
Ketika tahap \textit{specification} telah selesai, maka akan didapat pengetahuan mengenai domain yang masih belum terstruktur dengan baik. Pada tahap {Conceptualization}, pengetahuan
yang tidak terstruktur tersebut akan diatur dengan representasi yang tidak bergantung pada bahasa pemrograman dan lingkungan \textit{runtime}\cite{lopez1999building}. 
\subsection{Implementation}

\section{Desain Sistem}
Sebelum masuk pada fase pengembangan sistem, menentukan desain kasar sistem merupakan hal yang krusial. Hal tersebut disebabkan karena memahami komponen apa saja yang ada pada sistem dan relasi antar komponen dapat membantu menentukan arah pengembangan sistem. 
 
\subsection{Arsitektur Sistem}
\par
Gambar 3.1 menjelaskan desain arsitektur sistem dan relasi-relasi antar komponen.
\newline
\begin{figure}[h!]
    \centering
    \includegraphics[scale=0.3]{img/arsitektur_sistem.png}
    \caption{Gambaran komponen-komponen sistem dan relasinya}
    \label{fig:Gambar}
\end{figure}

\par
Komponen-komponen RS yang akan dikembangkan penyusun terdiri dari \textit{client}, \textit{server}, basis data dengan fail OWL, Google Maps API dam OpenWeatherMap API. 
\textit{Client} berkomunikasi dengan Google Maps API untuk menampilkan data lokasi pengguna dan tujuan wisata. Selanjutnya \textit{client} mengirimkan \textit{HTTP request} ke 
\textit{server} dengan menyertakan data lokasi dan preferensi pengguna. Ketika \textit{server} menerima \textit{HTTP request}, \textit{server} akan membalas dengan 
\textit{HTTP response} yang berisi daftar rekomendasi tujuan wisata.

\subsection{Model Ontology}

Gambar 3.2 adalah garis besar model \textit{ontology} yang merepresentasikan pengetahuan sistem yang akan dibangun:
\begin{figure}[h!]
    \centering
    \includegraphics[scale=0.6]{img/ontology.png}
    \caption{Gambar \textit{ontology} representasi pengetahuan RS}
    \label{fig:Gambar}
\end{figure}
\newline
Struktur \textit{ontology} dibangun berdasarkan asumsi bahwa tempat wisata yang dapat dikunjungi ketika hujan juga dapat dikunjungi ketika cuaca cerah, tetapi tidak berlaku sebaliknya. Strukturnya juga menyimpan informasi mengenai lokasi dan waktu buka tempat wisata.
 
\subsection{Flowchart Sistem}
Secara umum, aliran kerja sistem dijelaskan pada gambar 3.3 dibawah ini:
\begin{figure}[h!]
    \centering
    \includegraphics[scale=0.6]{img/flowchart-general.png}
    \caption{Gambaran aliran kerja sistem yang dikembangkan}
    \label{fig:Gambar}
\end{figure}
\textit{Flowchart} sistem pada gambar 3.3 dapat dijelaskan secara umum sebagai tiga kegiatan utama. Tiga kegiatan tersebut adalah mendapatkan alur mendapatkan informasi kontekstual dan preferensi tujuan wisata pengguna, melakukan \textit{reasoning} di \textit{web server} berdasar data yang masuk, dan mendapatkan rekomendasi tujuan wisata secara visual.
\section{Metodologi Pengembangan Sistem}

Tugas akhir ini akan disusun dengan melalui tahap-tahap berikut:
\begin{enumerate}

\item Desain Sistem
\par
Dalam tahap ini ada dua bagian, yaitu desain struktur \textit{ontology} dan pengembangan sistem.
\par
Tahap-tahap dalam mendesain struktur \textit{ontology} mengikuti \textit{Methontology} \cite{jones1998methodologies}, yaitu:
\begin{enumerate}
	\item Spesifikasi
	\item Akuisisi pengetahuan
	\item Konseptualisasi
	\item Integrasi
	\item Implementasi
	\item Evaluasi
	\item Dokumentasi
\end{enumerate} 
Berdasarkan lima komponen penting pada RS yang telah dijelaskan di gambar 3.1, berikut adalah langkah bagian pengembangan sistem yang akan ditempuh:
\begin{enumerate}
	\item Merancang GUI pada sisi \textit{client}.
	\item Merancang skema interaksi penggunaan aplikasi.
	\item Merancang format data yang dipertukarkan antara \textit{client} dan \textit{web server}.
	\item Merancang algoritma pada sisi \textit{client} dan \textit{web server}.
	\item Merancang \textit{query} untuk mengakuisisi pengetahuan pada data yang disimpan di \textit{SPARQL}.
\end{enumerate}

\item Pengumpulan Data Lokasi Wisata Bandung
\newline
Tujuan tahap ini adalah mendapatkan data lokasi wisata. Metode yang dapat dilakukan adalah:
\begin{enumerate}
	\item Observasi.
	Metode observasi dilakukan untuk mendapatkan data lokasi wisata yang tidak dikelola oleh pihak swasta.
	\item Wawancara pihak pengelola tempat wisata.
	Metode wawancara dilakukan untuk lokasi wisata yang dikelola oleh pihak swasta.
\end{enumerate}
\par
Tujuan dari tahap ini adalah mendapatkan data koordinat tujuan wisata di \textit{Google Maps}, analisis kondisi tujuan wisata pada cuaca cerah dan hujan, serta waktu ketika tujuan wisata dapat diakses.

\item Implementasi Sistem
\newline
Tahap implementasi meliputi:
\begin{enumerate}
	\item Pengembangan aplikasi Android.
	\item Pengembangan sisi \textit{server}.
	\item Implementasi komunikasi antar isi \textit{client} dan sisi \textit{server} serta \textit{client} dan \textit{Google Maps API}.
	\item Implementasi \textit{ontology} pada basis data \textit{SPARQL}.
\end{enumerate}
 
\item Pengujian
\newline
Tahap pengujian memiliki dua jenis pengujian:
\begin{enumerate}
	\item Pengujian utnuk menemukan \textit{bug}. Untuk menemukan \textit{bug}, sistem akan diuji coba dengan menggunakan skema \textit{white box}.
	\item Pengujian akurasi sistem. 
\end{enumerate}

\item Pembuatan Laporan
\newline 
Laporan akhir akan ditulis sesuai dengan kaidah dan ketetapan yang berlaku di institusi.
\end{enumerate}


%
\chapter{Hasil dan Pembahasan}

\par
Pengujian dilakukan dengan pakar dan pengguna biasa, sehingga skenario pengujian pun berbeda.
\section{Hasil Pengujian dengan Pakar Pariwisata}

\par
Dengan menggunakan skenario \ref{table:scenario}, berikut adalah hasil yang didapatkan dari pengujian dengan pakar:
\begin{center}
\small
\begin{longtable}{ |l|l|l|l|l|l| } 
\hline
\textbf{No} & \textbf{Lokasi pengguna} & \textbf{Waktu} & \textbf{Cuaca} & \textbf{No Uji} & \textbf{Presisi}\\
\hline
1	&	Alun-alun Cimahi	&	09:00	& bagus & 1 & 1\\
\hline
2	&	Alun-alun Cimahi	&	09:00	& kurang bagus & 1 & 1\\
\hline
3	&	Alun-alun Cimahi	&	09:00	& buruk & 1 & 1\\
\hline
4	&	Alun-alun Cimahi	&	13:00	& bagus & 2 & 1\\
\hline
5	&	Alun-alun Cimahi	&	13:00	& kurang bagus & 2 & 1\\
\hline
6	&	Alun-alun Cimahi	&	13:00	& buruk & 2 & 0.9\\
\hline
7	&	Alun-alun Cimahi	&	18:00	& bagus & 3 & 1\\
\hline
8	&	Alun-alun Cimahi	&	18:00	& kurang bagus & 3 & 1\\
\hline
9	&	Alun-alun Cimahi	&	18:00	& buruk & 3 & 0.7\\
\hline
10	&	Jalan Setiabudhi Bandung	&	09:00	& bagus & 1 & 1\\
\hline
11	&	Jalan Setiabudhi Bandung	&	09:00	& kurang bagus & 1 & 1 \\
\hline
12	&	Jalan Setiabudhi Bandung	&	09:00	& buruk & 1 & 0.9\\
\hline
13	&	Jalan Setiabudhi Bandung	&	13:00	& bagus & 2 & 1\\
\hline
14	&	Jalan Setiabudhi Bandung	&	13:00	& kurang bagus & 2 & 1\\
\hline
15	&	Jalan Setiabudhi Bandung	&	13:00	& buruk & 2 & 0.9\\
\hline
16	&	Jalan Setiabudhi Bandung	&	18:00	& bagus & 3 & 1\\
\hline
17	&	Jalan Setiabudhi Bandung	&	18:00	& kurang bagus & 3 & 0.9\\
\hline
18	&	Jalan Setiabudhi Bandung	&	18:00	& buruk & 3 & 0.7\\
\hline
19	&	Jalan Buah Batu	&	09:00	& bagus & 1 & 1\\
\hline
20	&	Jalan Buah Batu	&	09:00	& kurang bagus & 1 & 1\\
\hline
21	&	Jalan Buah Batu	&	09:00	& buruk & 1 & 0.9\\
\hline
22	&	Jalan Buah Batu	&	13:00	& bagus & 2 & 1\\
\hline
23	&	Jalan Buah Batu	&	13:00	& kurang bagus & 2 & 1\\
\hline
24	&	Jalan Buah Batu	&	13:00	& buruk & 2 & 0.8\\
\hline
25	&	Jalan Buah Batu	&	18:00	& bagus & 3 & 1\\
\hline
26	&	Jalan Buah Batu	&	18:00	& kurang bagus & 3 & 0.8\\
\hline
27	&	Jalan Buah Batu	&	18:00	& buruk & 3 & 0.6\\
\hline
\caption{Tabel skenario pengujian dengan pakar}
\label{table:result-1}
\end{longtable}
\end{center}

\begin{figure}[h!]
    \centering
    \includegraphics[scale=0.7]{img/precision_weather.png}
    \caption{Mean presisi sistem berdasar kondisi cuaca}
    \label{fig:p_weather}
\end{figure}


\begin{figure}[h!]
    \centering
    \includegraphics[scale=0.7]{img/precision_time.png}
    \caption{Mean presisi sistem berdasar waktu lokal pengujian}
    \label{fig:p_time}
\end{figure}

\section{Hasil Pengujian dengan Pengguna Biasa}
\begin{figure}[h!]
    \centering
    \includegraphics[scale=1]{img/accuracy-gender.png}
    \caption{Akurasi sistem berdasar jenis kelamin}
    \label{fig:accuracy-gender}
\end{figure}

\begin{figure}[h!]
    \centering
    \includegraphics[scale=0.7]{img/hasil_kuesioner.png}
    \caption{Hasil kuesioner pengguna aplikasi}
    \label{fig:quest}
\end{figure}
%
\chapter{Kesimpulan}
\section{Kesimpulan}
\begin{enumerate}
	\item Presisi rekomendasi destinasi wisata menjadi turun jika sistem digunakan pada malam hari. 
	
\end{enumerate}
\section{Saran}
\par
Berikut adalah saran mengenai hal-hal yang bisa dikembangkan dari tugas akhir ini:
\begin{enumerate}
	\item Menambah masukan \textit{contextual factor} seperti suasana hati pengguna dan keramaian jalan menuju destinasi wisata.
	\item Menggunakan \textit{navigation by proposing} yang digabung dengan \textit{contexttual factor} sehingga sistem lebih intuitif
	dengan masukan pengguna.
\end{enumerate}

%
\cleardoublepage
\addcontentsline{toc}{chapter}{Daftar Pustaka}
\bibliographystyle{acm} %harvard style
\bibliography{References.bib}
%
%\pagebreak
\cleardoublepage
\addcontentsline{toc}{chapter}{Lampiran}
\include{Lampiran}
\end{document}
	