\chapter*{Abstract}

\par
Tourism sector shows increasing contribution for country revenue. One of key factor is how convenient to get Indonesia tourism information through internet service such as TripAdvisor
and Instagram. However, those commercial internet services are not yet able to provide personalized tourism destination
recommendation.
\par
Recommender systems are software tools and technique providing suggestions for item recommendation
based on user's needs. To provide item recommendation as system output, indeed suitable system input is needed. One of technique that can be implemented
to get recommender system input is context-aware.
\par
This undergraduate thesis is written through recommender system development and testing which has context-aware properties. This recommender
system is making use of user location, weather around tourism destinations, local time and user preference towards 
tourism destination category. Those system inputs are processed using knowledge-based ontological reasoning and recommender system
will propose tourism destination recommendation.

\vspace{0.5 cm}
\begin{flushleft}
{\textbf{Keywords:} tourism destination, recommender system, context-aware, knowledge-based, ontology}
\end{flushleft}