\chapter*{Abstrak}

\par
Sektor pariwisata menunjukkan tren kenaikan untuk sumbangan devisa negara. Salah satu faktor kenaikan tersebut adalah semakin mudahnya mendapatkan informasi tujuan wisata di Indonesia 
melalui layanan Internet yang ada seperti TripAdvisor dan Instagram. Namun hingga saat ini, masih belum ada layanan Internet komersil yang mampu memberikan rekomendasi tujuan wisata 
yang terpersonalisasi. 
\par
\textit{Recommender system} adalah teknik dan kakas perangkat lunak yang digunakan untuk menyediakan rekomendasi item yang sesuai dengan kebutuhan pengguna. 
Untuk dapat menyediakan rekomendasi item sebagai keluaran sistem, tentu dibutuhkan masukan yang sesuai. 
Salah satu teknik yang dapat diterapkan untuk memperoleh masukan \textit{recommender system} adalah \textit{context-aware}. \textit{Recommender system} dapat bersifat \textit{context-aware} dengan memanfaatkan informasi kontekstual pengguna seperti lokasi dan cuaca.
\par
Tugas akhir ini disusun melalui pengembangan dan pengujian \textit{recommender system} yang bersifat \textit{context-aware} dengan 
memanfaatkan masukan berupa lokasi pengguna, cuaca di sekitar tujuan wisata, waktu lokal dan preferensi pengguna terhadap kategori wisata, 
kemudian diproses dengan melakukan \textit{reasoning} pada salah satu model objek dalam \textit{knowledge-based}, yaitu \textit{ontology}, sehingga menghasilkan rekomendasi tujuan wisata.

\vspace{0.5 cm}
\begin{flushleft}
{\textbf{Kata Kunci:} tujuan wisata, \textit{recommender system}, \textit{context-aware}, \textit{knowledge-based}, \textit{ontology}}
\end{flushleft}